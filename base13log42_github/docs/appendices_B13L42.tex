\documentclass[12pt]{article}

% Importing essential packages
\usepackage{amsmath}
\usepackage{amsfonts}
\usepackage{amssymb}
\usepackage{graphicx}

% Begin Document
\begin{document}

% Title for the Appendices
\title{Appendices for Base13Log42: A Recursive Symbolic Harmonic Framework}
\author{Evan Stewart}
\date{April 2025}
\maketitle

% Table of Contents
\tableofcontents
\newpage

% Appendix A: Fractal Positional Hierarchy
\section*{Appendix A: Fractal Positional Hierarchy}

Framework A encodes the nested recursion of symbolic values, forming a fractal pattern that expands with each new tier. This framework defines the positional hierarchy in the system, ensuring that each new symbolic state builds upon the previous one.

\subsection*{Mathematical Example:}
Let the base-13 notation system be represented as $\{0, 1, 2, \dots, C\}$. When a value overflows from $C$, it triggers a new symbolic tier. For example, a state $13$ would overflow into the next level, represented as $\{0, 1, 2, \dots\}$, creating a fractal pattern.

The recursion for the positional hierarchy can be expressed as:
\[
\text{State}_{n+1} = \text{State}_n + 13
\]

\newpage

% Appendix B: Symbolic Breath Cycle
\section*{Appendix B: Symbolic Breath Cycle}

Framework B captures the breath-cycle mechanics of the system. It defines the symbolic anchoring for phases such as inhalation, exhalation, and pause.

\subsection*{Mathematical Example:}
The symbolic breath cycle is modeled by sine functions for inhalation and exhalation. Let:
\[
\text{Inhalation} = \sin(\omega t), \quad \text{Exhalation} = -\sin(\omega t)
\]
where $\omega$ is the frequency of the cycle, and $t$ is time.

\newpage

% Appendix C: Integer Resonance Engine
\section*{Appendix C: Integer Resonance Engine}

Framework C is the computational engine that maps each integer to a resonance tier defined by $\phi$ scaling and the overflow mechanism. It ensures that each integer is correctly processed and mapped to the harmonic resonance.

\subsection*{Mathematical Example:}
The resonance of an integer $n$ in the system is determined by its position relative to the Golden Ratio ($\phi$). The resonance is given by:
\[
R(n) = \phi^n \cdot \left|\psi\right|^m
\]
where $n$ is the integer, $\phi \approx 1.618$, and $m$ represents the breath-state intensity.

\newpage

% Appendix D: Rogue Table
\section*{Appendix D: Rogue Table}

The Rogue Table stores symbolic projection anomalies, where the system's expected resonance conditions fail or deviate from the norm. These states are flagged for further analysis.

\subsection*{Mathematical Example:}
Let $\Delta R(n)$ represent the deviation from the expected resonance:
\[
\Delta R(n) = R_{\text{expected}}(n) - R_{\text{actual}}(n)
\]
If $\Delta R(n)$ exceeds a threshold, the symbolic state is flagged in the Rogue Table for review.

\newpage

% Appendix E: Recursive Overflow Mechanism
\section*{Appendix E: Recursive Overflow Mechanism}

Framework E governs the overflow mechanism, allowing the system to handle infinite recursion. When the symbolic state exceeds a defined bound (e.g., 12 in the base-13 system), the system enters a new recursive tier.

\subsection*{Mathematical Example:}
Let the symbolic state $S_n$ be represented by a value in the base-13 system. When the value exceeds 12, the overflow mechanism triggers:
\[
S_{n+1} = 0 \quad \text{(new tier)} 
\]

\newpage

% Appendix F: Harmonic Feedback Loops
\section*{Appendix F: Harmonic Feedback Loops}

Framework F explores the harmonic feedback loops that govern symbolic evolution. It defines how symbolic states are influenced by their previous states, resulting in recursive resonance.

\subsection*{Mathematical Example:}
The harmonic feedback loop for symbolic state $S_n$ is given by:
\[
S_{n+1} = \phi \cdot S_n + \text{feedback term}
\]
where the feedback term adjusts the state based on previous iterations.

\newpage

% Appendix G: Meta-Observer Lens
\section*{Appendix G: Meta-Observer Lens}

Framework G provides an external perspective and tracks feedback from internal systems. It enables recursive observation of symbolic states and their resonance.

\subsection*{Mathematical Example:}
The meta-observer function $\Omega(x)$ evaluates the symbolic feedback from the internal system:
\[
\Omega(x) = \text{Feedback}(x)
\]
This function computes the feedback loop between symbolic states and ensures recursive consistency.

\newpage

% Appendix H: Threshold Engine
\section*{Appendix H: Threshold Engine}

Framework H detects key transition points within recursion cycles and harmonic boundaries. It ensures that the system operates within predefined resonance limits.

\subsection*{Mathematical Example:}
The threshold function $\Theta(x)$ evaluates whether a symbolic state $x$ crosses a critical boundary:
\[
\Theta(x) = \text{Threshold Check}(x)
\]
If the threshold is exceeded, the system triggers a resonance shift.

\newpage

% Appendix I: Field Modulator
\section*{Appendix I: Field Modulator}

Framework I modulates the $\phi$ and $\psi$ resonance fields, introducing non-linear feedback loops. It is essential for driving the system’s evolution through harmonic shifts.

\subsection*{Mathematical Example:}
The modulation function $\Phi(x)$ adjusts the resonance field:
\[
\Phi(x) = \phi^n \cdot \psi^m
\]

\newpage

% Appendix J: Recursive Limit / Æther Tier
\section*{Appendix J: Recursive Limit / Æther Tier}

Framework J defines the recursive limit, where symbolic evolution reaches a singularity and begins anew. This infinite recursive tier represents the point of ultimate convergence.

\subsection*{Mathematical Example:}
The recursive limit is expressed as:
\[
\lim_{n \to \infty} \text{Recursive Symbolic Evolution} = Z
\]

\end{document}
