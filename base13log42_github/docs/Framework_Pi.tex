\section*{Framework $\Pi$ — Unified Mapping Field}

\subsection*{Core Purpose}

Framework $\Pi$ defines the \textbf{resonance-space geometry} of Base13Log42. It introduces a coordinate-based structure that embeds symbolic states, breath-phase dynamics, and harmonic values into a multidimensional field. Within this field, symbolic resonance becomes a spatial function, allowing transformations, inversions, and mappings to be defined as geometric operations.

This framework enables visual symmetry, state tracking, symbolic proximity analysis, and resonance-space navigation.

\subsection*{Coordinate Embedding of Symbolic States}

Each symbolic state $x$ within shell $k$ is mapped into a coordinate space $\mathbb{R}^n$ via:

\[
\Pi(x) = \left( b_k(x),\ R_k(x),\ \theta_k(x) \right)
\in [0,1] \times [0,1] \times [0,2\pi]
\]

Where:
\begin{itemize}
  \item $b_k(x)$ is the breath-state intensity (Framework B$'$)
  \item $R_k(x)$ is the resonance value (Framework C)
  \item $\theta_k(x)$ is the symbolic phase (Framework D)
\end{itemize}

This constructs a 3D harmonic space where every symbolic state is a moving point on a resonant manifold.

\subsection*{Resonance Distance Function}

To compare symbolic states in the mapping field, define the \textbf{resonance distance} between states $x$ and $y$:

\[
d_\Pi(x, y) = \sqrt{
(b_k(x) - b_k(y))^2 +
(R_k(x) - R_k(y))^2 +
\left( \frac{\theta_k(x) - \theta_k(y)}{2\pi} \right)^2
}
\]

This defines a metric topology over the symbolic logic system, where resonance compatibility corresponds to geometric proximity.

\subsection*{Mapping Transformations}

Framework $\Pi$ allows transformation functions to be defined over the mapping field. Examples include:

\begin{itemize}
  \item \textbf{Reflection} — symbolic inversion across a harmonic axis
  \item \textbf{Rotation} — phase shift via angle addition
  \item \textbf{Translation} — breath-state advance or compression
\end{itemize}

Transformations preserve or adjust resonance, allowing symbolic dynamics to be modeled like vector flows in a harmonic space.

\subsection*{Mapping Overlay Functions}

Let $\mathcal{F}_\Pi$ be a field function over $\mathbb{R}^3$ representing symbolic pressure, intensity, or overflow buildup. This enables contour plotting of system stress points:

\[
\mathcal{F}_\Pi(b, R, \theta) \rightarrow \mathbb{R}
\]

Useful for visual diagnostics, symbolic field resonance monitoring, or overflow anticipation.

\subsection*{Framework $\Pi$ Summary}

\begin{center}
\begin{tabular}{|l|l|}
\hline
\textbf{Category} & \textbf{Description} \\
\hline
Name & Unified Mapping Field \\
Mapping Function & $\Pi(x) = (b_k(x), R_k(x), \theta_k(x))$ \\
Resonance Distance & $d_\Pi(x,y)$: Euclidean metric over symbolic coordinates \\
Transformations & Rotation, reflection, translation of symbolic dynamics \\
Overlay Fields & $\mathcal{F}_\Pi$: pressure, overflow, or resonance strength \\
Use Cases & Symbol tracking, resonance symmetry, harmonic diagnostics \\
Depends On & B$'$, C, D \\
Feeds Into & $\Sigma$, $\Phi$, $\infty$ \\
\hline
\end{tabular}
\end{center}

\subsection*{Interpretation}

Framework $\Pi$ reveals that symbolic logic lives not only in numbers or glyphs, but in \textbf{harmonic space}. Every transformation, resonance shift, or overflow burst leaves a geometric trace. By embedding symbolic states in a coordinate field, the system becomes navigable — a symbolic cartography of logic in motion.

