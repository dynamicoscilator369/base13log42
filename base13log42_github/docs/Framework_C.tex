\section*{Framework C — Resonance Logic}

\subsection*{Core Purpose}

Framework C introduces the principle of \textbf{resonance logic}, a non-Boolean method for determining the validity of symbolic transformations based on harmonic alignment. Logical operations are not evaluated in terms of ``true'' or ``false,'' but in terms of \textbf{resonant compatibility} between symbolic states.

This framework enables Base13Log42 to function as a harmonic logic engine — where state transitions are only allowed if they align with the system’s recursive frequency model.

\subsection*{Resonance Value of Symbolic State}

Each symbolic state $x$ is associated with a \textbf{resonance value} $R(x)$, which quantifies its alignment with the internal harmonic structure of its shell.

Let:
\[
R_k(x) = \sin^2\left( \pi \cdot b_k(x) \right)
\]

where $b_k(x)$ is the breath-state intensity defined in Framework B$'$:
\[
b_k(x) = \frac{x - S_{k-1}'}{S_k' - S_{k-1}'}
\]

$R_k(x)$ ranges from $0$ (symbolic rest) to $1$ (peak resonance), with a smooth sinusoidal profile.

\subsection*{Resonant Compatibility Function}

To determine whether two symbolic states can interact or transition, define the \textbf{resonance compatibility function}:

\[
\mathcal{R}(x, y) = 1 - \left| R_k(x) - R_k(y) \right|
\]

This measures harmonic closeness between two symbolic states in the same shell. Transitions are permitted if:

\[
\mathcal{R}(x, y) \geq \epsilon
\]

where $\epsilon$ is a system-defined threshold (e.g., $0.9$ for high harmonic purity).

\subsection*{Tier-Based Resonance (Resonance Tier)}

Each shell $k$ may define a \textbf{resonance tier} $T_k$ — a quantized zone within the sinusoidal resonance curve. Symbolic states are assigned a tier number based on their $b_k(x)$ value. For example:

\begin{center}
\begin{tabular}{|c|c|}
\hline
\textbf{Tier} & \textbf{Breath Intensity Interval ($b_k(x)$)} \\
\hline
1 & $[0.00,\ 0.25]$ \\
2 & $[0.25,\ 0.50]$ \\
3 & $[0.50,\ 0.75]$ \\
4 & $[0.75,\ 1.00]$ \\
\hline
\end{tabular}
\end{center}

Transitions across tiers are weighted lower unless guided by external overflow logic (Framework $\Lambda$).

\subsection*{Expected Resonance Conditions}

Each shell has a set of \emph{expected resonance conditions} — ideal symbolic configurations whose resonance values align across the recursive hierarchy. Deviation from these conditions introduces ``dissonance,'' which may trigger symbolic overflow, reset, or inversion.

\subsection*{Resonance Logic Rule (Symbolic Validation)}

A symbolic transformation $x \rightarrow y$ is \textbf{valid} under resonance logic if:

\[
\mathcal{R}(x, y) \geq \epsilon \quad \text{and} \quad T_k(x) = T_k(y)
\]

Otherwise, the system must route through a resonance-correcting operation (e.g., via $\dot{0}$ reset, inversion, or overflow channel).

\subsection*{Framework C Summary}

\begin{center}
\begin{tabular}{|l|l|}
\hline
\textbf{Category} & \textbf{Description} \\
\hline
Name & Resonance Logic \\
Resonance Value & $R_k(x) = \sin^2(\pi b_k(x))$ \\
Compatibility Function & $\mathcal{R}(x, y) = 1 - |R_k(x) - R_k(y)|$ \\
Resonance Tier & Based on $b_k(x)$ value intervals \\
Validation Rule & $\mathcal{R}(x, y) \geq \epsilon$ and $T_k(x) = T_k(y)$ \\
Failure Condition & Transition blocked or rerouted via overflow/reset \\
Depends On & Frameworks A (Position), B$'$ (Breath-State) \\
Feeds Into & $\Lambda$ (Overflow), $\Sigma$ (Recursive Feedback) \\
\hline
\end{tabular}
\end{center}

\subsection*{Interpretation}

Framework C defines the logic \emph{pulse} of the system. Instead of binary gates, Base13Log42 uses harmonic fields to validate logic transitions. This enables logic to behave more like music — relational, recursive, and rhythmically contextual — rather than strict computation.
