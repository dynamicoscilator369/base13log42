
`symbolic_clock.md`

```markdown
# Symbolic Clock in Base13Log42

## Overview
The Symbolic Clock represents a timed cycle in the Base13Log42 system, controlling how symbolic states progress or pause over discrete time intervals. It ties together the base-13 breath concepts, sine-wave cycles, and resonance thresholds.

## Key Points

1. **Breath-Cycle Synchronization**:  
   Each symbolic state moves forward only when the breath-cycle sine wave is in a “positive” phase. This ensures the system evolves in cycles—like inhalation/exhalation.

2. **Discrete Time Steps**:  
   - A `t` parameter (often a real number) steps forward in increments (`dt`).
   - `breath_cycle(t)` determines whether we move to the next symbol or stay still.

3. **Overflow and Reset**:  
   - When the symbol `C` transitions to `one`, it signifies an overflow event.
   - This can trigger special feedback (e.g., doubling resonance in the `feedback_system`).

## Example Flow

```text
t=0.0   → state=one    → breath_cycle(0.0)=0  → stays at one
t=0.1   → state=one    → breath_cycle(0.1)=0.587>0 → moves to two
t=0.2   → state=two    → breath_cycle(0.2)=0.951>0 → moves to three
…
```

When `breath_cycle(t)` crosses zero from positive to negative, the system stalls. Once it goes back above zero, the clock “ticks” to the next symbol.

## Future Ideas

- **Multiple Clocks**: Some expansions might involve multiple overlapping clocks (or multiple breath-cycles), each controlling different aspects of state transitions.
- **Variable Threshold**: Instead of checking `>0`, you might only advance the state if `breath_cycle(t) > 0.5`, giving a narrower window for transitions.
- **Adaptive dt**: Time steps can be dynamic, adjusting based on internal feedback or external triggers.

```

---

## `observer_chains.md`

```markdown
# Observer Chains in Base13Log42

## Introduction
Framework Ω (the Observer Lens) introduces the concept of _observer chains_: sequences in which symbolic states are repeatedly examined and possibly altered by an external “observer.” This can model self-reference, measurement, or feedback from outside the system.

## Core Ideas

1. **Observer Predicate**  
   We often define something like:
   ```lean
   def observer (x : SymbolicState) : Prop :=
     -- some condition or lens function
   ```
   This can be read as “the system is being observed at state `x`,” or “`x` is flagged for observation.”

2. **Self-Reference / Gödel-Style**  
   Observer chains allow the system to refer to itself. For example, an observer might detect if `C → one` overflow just happened and then alter the resonance or forcibly reset a sub-cycle. This introduces a self-referential loop where the system modifies itself upon observation.

3. **Recursive Chains**  
   - An observer chain might look like:
     ```text
     state0 --(observed)--> state1 --(observed)--> state2 -- ...
     ```
   - Each link in the chain is triggered by “did we observe the last state?” or “did the observer condition hold?”

## Example in Lean

Here’s a _conceptual_ snippet:
```lean
def is_overflow_event (s₁ s₂ : SymbolicState) : bool :=
  (s₁ = SymbolicState.C ∧ s₂ = SymbolicState.one)

def observer_update (prev : SymbolicState) (cur : SymbolicState) (res : ℝ) : (SymbolicState × ℝ) :=
  if is_overflow_event prev cur then (cur, res * 2)
  else (cur, res)
```

## Use Cases

- **Measuring Convergence**: Observers might track how quickly the system’s resonance converges, halting if a certain threshold is reached.
- **Interrupts & Triggers**: Certain events (like overflow) can “notify” an observer, leading to additional logic or external code.

```

---

## `rogue_table.md`

```markdown
# Rogue Table (Framework D)

## Purpose
Within Base13Log42, _Rogue Table_ refers to the set of states where the expected resonance or symbolic cycle pattern “breaks” or deviates. These anomalies can happen due to rounding errors, near-misses in the φ-based alignment, or special overflow conditions not accounted for in the standard cycle rules.

## Definition
**Rogue states**:
- States that appear out of alignment with the standard 13-phase cycle (e.g., the system fails to reset properly, generating `1.0̇.3` unexpectedly).
- States that yield an unexpected `resonance` value (like negative or extremely large) due to accumulative feedback.

A table (the “Rogue Table”) is typically used to list these states for further analysis or debugging.

## Example Criteria

1. **Ghost Overflow**  
   If `next_symbol(C)` = `A` instead of `one`, the system has a mismatch. This state is recorded in the Rogue Table with a note, e.g.:
   ```plaintext
   State: (C -> A)
   Error: Overflow mismatch
   ```

2. **Near Zero Resonance**  
   If `resonance(x, n)` becomes nearly zero for an unexpected reason (like `n` = 0 or floating errors), it can be flagged as a “ghost” or “Rogue” entry.

## Lean Sketch

One might define a function:
```lean
def rogue_condition (st1 st2 : SymbolicState) (res : ℝ) : bool :=
  (st1 = SymbolicState.C ∧ st2 ≠ SymbolicState.one) ∨
  (res < 0.001)  -- near zero, suspicious
```
And store these anomalies in a “Rogue Table” structure for analysis.

## Applications
- **Debugging & Diagnostics**: Helps track and fix undesired states.
- **Extended Framework**: Future expansions might incorporate automatic corrections or alternate transitions for rogue states (e.g., forcing a re-sync with the base cycle).

```

---

**How to Download**  
If you need these as files, simply copy each code block (between the triple backticks) into its respective `*.md` file:

- `symbolic_clock.md`
- `observer_chains.md`
- `rogue_table.md`

Then commit them to your repo. Once committed to GitHub (or your chosen version control), they’ll be individually downloadable as `.md` files. If you’re on a local machine, just save them as normal text files with a `.md` extension. 

That’s all! If you need further changes or expansions, let me know.